\documentclass[10pt,twocolumn,letterpaper]{article}

\usepackage{cvpr}
\usepackage{times}
\usepackage{epsfig}
\usepackage{graphicx}
\usepackage{amsmath}
\usepackage{amssymb}

\usepackage{url}

% Include other packages here, before hyperref.

% If you comment hyperref and then uncomment it, you should delete
% egpaper.aux before re-running latex.  (Or just hit 'q' on the first latex
% run, let it finish, and you should be clear).
%\usepackage[pagebackref=true,breaklinks=true,letterpaper=true,colorlinks,bookmarks=false]{hyperref}

\cvprfinalcopy % *** Uncomment this line for the final submission

\def\cvprPaperID{****} % *** Enter the CVPR Paper ID here
\def\httilde{\mbox{\tt\raisebox{-.5ex}{\symbol{126}}}}

% Pages are numbered in submission mode, and unnumbered in camera-ready
\ifcvprfinal\pagestyle{empty}\fi
\begin{document}

%%%%%%%%% TITLE
\title{Image classification analysis with Bag of Word model}

\author{Lorenzo Cioni\\
{\tt\small lore.cioni@gmail.com}
% For a paper whose authors are all at the same institution,
% omit the following lines up until the closing ``}''.
% Additional authors and addresses can be added with ``\and'',
% just like the second author.
% To save space, use either the email address or home page, not both
\and
Saverio Meucci\\
{\tt\small smeucci91@gmail.com}
}

\maketitle
\thispagestyle{empty}

%%%%%%%%% ABSTRACT
\begin{abstract}
  Abstract
\end{abstract}

%%%%%%%%% BODY TEXT
{\footnotesize \noindent\textbf{Keywords} - Bag of Words (BoW), SIFT, SVM, Nearest Neighbours, Hard-Soft Assignment}.

\section{Introduction}

Brief introduction

\vspace{3cm}

%-------------------------------------------------------------------------
\section{Feature extraction}

Feature extraction step

%-------------------------------------------------------------------------
\section{Image representation}

Image representation in BoW

\subsection{Hard assignment}

\subsection{Soft assignment}

\section{Image classification}

\subsection{Nearest Neighbours}

\subsection{Support Vector Machines (SVM)}

\subsubsection{Kernels}




Image classification

\section{Conclusion}

Conclusions

\end{document}
