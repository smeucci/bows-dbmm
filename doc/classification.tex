\section{Image classification}

After the feature quantization step, we can proceed to the classification step. For that purpose we evaluated the performance of two type of classifier: \emph{Nearest Neighbor} (NN) and \emph{Support Vector Machine} (SVM).

\paragraph{Nearest Neighbor}

Nearest Neighbor classifies an image in base of the nearest train example. Those classifiers differentiate each other for distance measure used. In this case only Euclidean distance (L2) and $\chi^2$ distance were been used.

\paragraph{Support Vector Machine}

\emph{Support Vector Machines} classifies an image identifying a separation surface on feature space. Those surface are evaluated using different kernel functions. The kernels used in our experiments were:
\begin{itemize}
\item \emph{Linear kernel}
\begin{equation}
K(x, z) = <x, z>
\end{equation}
\item \emph{Histogram Intersection kernel}
\begin{equation}
K(x, z) = \sum_{i = 1}^{n} \min (x_i, z_i)
\end{equation}
\item \emph{$\chi^2$ kernel}
\begin{equation}
K(x, z) = \exp (-\gamma \sum_{i = 1}^{n} \frac{(x_i - z_i)^2}{x_i + z_i})
\end{equation}
\item \emph{RBF kernel}
\begin{equation}
K(x, z) = \exp (- \gamma ||x_i - z_i||)
\end{equation}
\end{itemize}
SVM classifier depends on some parameters: C and $\gamma$ values are obtained with 5-fold cross-validation.