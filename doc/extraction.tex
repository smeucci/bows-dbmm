\section{Feature extraction}

Generally, global features are not very robust as a change in part of the image may cause it to fail as it will effect the resulting descriptor, thus they are not very good for image classification.

For this reason, local features are used, which descriptors describes a patch within an image. Instead of having a single descriptor for the entire image, with local features we have a set of descriptors, each describing a particular local features of the image.

To extract local features descriptors, we first need to detect the \emph{key-points} in an image.
Key-points can be extracted in three different ways: \emph{sparse}, e.g. using the Harris-Laplace key-points detector; \emph{dense}, key-points are taken on a grid of overlapped patches; \emph{multi-scale dense sampling}, same as dense but taken on varying the image resolution.

Once we have the image key-points, we can proceed to compute the local features descriptors, one for each key-points. In literature, many descriptors were proposed; the one used in this study is the SIFT descriptor, a spatial histogram of the image gradients in characterizing the appearance of each key-point, resulting in a vector of 128 elements. 

SIFT descriptors are invariant to scale, to affine transformation, and to a certain degree to prospective transformation.