\section{Feature extraction}

%which describes the features employed and their properties.

%- features globali non precise per la classificazione
%- uso features locali; come le estraggo?
%- sparse, uso un keypoints detector come harris laplace
%- dense, prendo keypoints uniformi sull'immagine
%- Multi-Scale Dense Sampling, prendo keypoints densi a scale diverse dell'immagine.
%- estratti i keypoints uso sift descriptor per descrivere ogni features locale dell'immagine
%- perchè sift? proprietà di invarianza, cosa prendono in considerazione di un'immagine.

Global features are not very good for image classification, because describing an image as its color or edge distribution is not indicating of a specific class of objects.

For this reason, local features were proposed. A local features of an image is a salient point in an image, like an edge point. Instead of having a single descriptor for the entire image, with local features we have a set of descriptors, each describing a particular local features of the image.

To extract local features descriptors, we first need to detect the key-points in an image.
Key-points can be extracted in an image in three different ways: sparse, using the Harris-Laplace key-points detector; dense, key-points are taken on a grid of overlapped patches; multi-scale dense sampling, same as dense but taken varying the resolution of the image.

Once we have the key-points of an image, we can proceed to computed the local features descriptors, one for each key-points. In literature, many descriptors were proposed; the one used in this study is the SIFT descriptor. For each local features it is computed a SIFT descriptor, a vector of 128 elements, that takes into account orientation. SIFT descriptors are invariant to scale, to affine transformation, and to a certain degree to prospective transformation.